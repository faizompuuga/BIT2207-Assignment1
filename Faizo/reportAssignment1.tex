\documentclass[12pt]{report}

\title{\textbf{A REPORT ON AUTOMATED SERVER COOLING SYSTEMS}}
\author{MPUUGA FAIZO 14/U/9319/EVE 214004093}
\begin{document}
\maketitle

\chapter{}
\section{Introduction} 
\paragraph{An automated server cooling system is a system that controls and monitors temperature of a server and in a server room automatically with less or no intervention of human labour.
Automation refers to self-operation without human intervention.
Cooling refers to lowering of increasing or already high temperatures, in this case, produced by the technology equipment in the server.}


 \section{Background}
\paragraph{According to Legacy cooling at the end of Raised floor, for decades, server rooms used raised floor systems to deliver cold air to servers with a main aim to deliver cold air where it is needed with very little effort by simply swapping a solid tile for a perforated tile.}

\paragraph{Looking at Layout and Heat ejection options, raised floor was okay with low-densities but failed to meet demands of increasing heat densities and efficiency, hence this called upon people to employ best practices like hot and cold Aisles, ceiling return plenums, raised floor management to improve performance in raised floor environments.}

 \subsection{Server cooling styles}
\paragraph{These are the most commonly known and used cooling styles used to cool server temperatures in organisations today and they include;}

\begin{enumerate}
 \item  \textbf{Air circulation:} Is a method of dissipating heat which works by expanding the surface area or increasing the flow of air over the object(servers) to be cooled.
\item \textbf{Direct-to-chip, hot water liquid cooling:} This style looks at waste heat reuse to efficiently and sustainably develop data centers. The use of this style makes it possible to transfer a significant amount of heat back to usage besides, it notably reduces the power consumption.
\item \textbf{Cooling by immersion:} This is a computer cooling practice by which computer components or servers are submerged in a thermally but not electrically conductive liquid.
\end{enumerate}

\subsection{How the system works}
The automated server cooling system works in a way that it has sensors that monitor temperature levels around the server and hence has the capability to trigger predefined mechanism against the current situation. In case of high temperatures, the system automatically turns on ACs, opens up ventilations to cool the temperature. For extremely high temperatures, the system is able to turn on alarms to alert a severe situation in the server room, thereby calling for human intervention to settle the problem.

\section{Conclusion}
There are a number of independent inbuilt cooling systems in the servers like air circulation, server room cooling gadgets like fun, Air conditioners, humidifier and a variety of standard systems for cooling like Aisles, containment strategies and raising the operating temperature of Air Handling Units (AHUs). Therefore a system that combines the entire cooling components and setting up a network of such components was called for hence the automated server cooling system which is highly desirable by any company or organization with a server room.

\section{References}
\begin{enumerate}
\item A look at data center cooling technologies (2015, July 30th). Retrieved from http://journal.uptimeinstitute.com/a-look-at-data-center-cooling-technology/
\item Server room problems for business to consider, Data centers by Tom Collins.
\end{enumerate}

\end{document}
